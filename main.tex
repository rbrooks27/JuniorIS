% This is a template for your written document.
%
% To compile using latexmk on the command line, run the following: 
% latexmk -pdf main.tex

\documentclass[12pt]{article}
\usepackage{setspace}
\singlespace
\usepackage[left=1in,right=1in,top=1in,bottom=1in]{geometry}

\title{\textbf{Multistep Synthesis Tool for Organic Chemistry Education}}
\author{Raelyn Brooks}

\begin{document}

\maketitle

\section{Introduction}
\indent
Organic synthesis—the design of reaction sequences to build complex molecules from simpler precursors—remains one of the most intellectually demanding areas of chemistry.
Unlike single-step reactions that can be memorized and applied in isolation, multistep synthesis requires strategic planning, backward reasoning, and the coordination of many interacting chemical transformations.
For students, this often feels like solving a “black box” puzzle: they understand individual reactions but struggle to combine them into coherent pathways.
National failure rates for organic chemistry courses frequently exceed 50 percent, underscoring the widespread difficulty of mastering this skill.
\\
\indent
n both research and education, the traditional static depiction of reactions in textbooks fails to reflect the underlying algorithmic logic of multistep synthesis.
Designing a synthetic route involves recursive problem solving (retrosynthesis), constraint management (functional group compatibility and selectivity), and temporary state control (protecting groups and stereochemistry)—all of which resemble computational planning problems.
However, existing instructional tools rarely make this algorithmic structure explicit, limiting learners’ ability to see synthesis as a structured, rule-driven process.
\\
\indent
This project addresses that gap by developing an interactive visualization pipeline that encodes reactions using the SMIRKS formalism and renders them using the RDKit cheminformatics library.
SMIRKS provides a symbolic language for reaction rules, enabling computational parsing, pattern recognition, and database integration.
RDKit then transforms these symbolic definitions into visual molecular diagrams, allowing entire multistep pathways to be represented as connected, interpretable reaction networks.
\\
\indent
By leveraging these technologies, the project contributes to educational cheminformatics through a tool that:
\begin{enumerate}
    \item Automates the rendering of multistep synthesis pathways from reaction datasets.
    \item isualizes reaction logic in a way that emphasizes decision trees and constraint satisfaction, mirroring algorithmic reasoning.
    \item Bridges chemical and computational representations, making synthesis more accessible for both chemists and computer scientists.
\end{enumerate}
\indent
Ultimately, this work positions multistep synthesis not just as a chemical skill but as a computationally tractable process, paving the way for more effective teaching tools, database-driven synthesis planning, and algorithmic analysis in organic chemistry.
\\


\section{Background}
\indent
In organic chemistry, multistep synthesis refers to the process of transforming simple molecular starting materials into complex target molecules through a series of discrete, logically ordered chemical reactions.
Each reaction step modifies the structure of a molecule, typically by altering functional groups—specific arrangements of atoms that dictate chemical behavior.
From a computer science perspective, these functional groups can be viewed as data types or states: just as an integer might be cast into a string or an array into a list, a hydroxyl group (–OH) might be converted into a halide (–Cl), or an alkene might be transformed into an alcohol. 
Each synthetic step is thus a state transition in a well-defined, though highly constrained, chemical state machine.
\\
\indent
Designing a synthetic pathway is not unlike algorithm design, where chemists aim to find an efficient and feasible sequence of operations to reach a target state.
Here, the “output” is a desired molecule, and the “input” is a set of commercially available or easily accessible starting materials. 
The multistep aspect arises because direct, one-step conversions are rarely possible; instead, chemists must plan a sequence of reactions, each altering the molecular structure in a controlled manner.
This is analogous to chaining multiple functions in a program to transform an initial data structure into a final desired format.
\\
\indent
A central algorithmic strategy in synthesis planning is retrosynthesis, introduced by E.J. Corey, organic chemist and 1990 Chemistry Nobel Prize Winner.
Retrosynthesis involves working backward from the target molecule to break it down recursively into simpler precursors.
This is strikingly similar to goal decomposition in AI planning or backtracking algorithms in search problems.
Starting from the target, chemists iteratively identify strategic bonds to “disconnect,” yielding simpler molecules that could plausibly be converted into the target in one step.
This continues until reaching molecules that are known or easily obtainable starting materials.
In computational terms, retrosynthesis resembles a reverse search through a tree or graph, where each node represents a molecular state and each edge represents a known chemical transformation.
\\
\indent
During synthesis planning, selectivity and functional group compatibility act as constraints, analogous to conditional logic in code or constraint satisfaction problems (CSPs) in AI.
For example, a reaction designed to modify an alcohol group might also inadvertently react with a nearby amine group, producing unwanted side products.
Chemists must therefore choose reaction sequences that respect these constraints, ensuring that each transformation occurs at the correct site and does not disrupt other parts of the molecule.
This is akin to managing function side effects in programming—ensuring that an operation modifies only what it is intended to, without corrupting unrelated data.
\\
\indent
An additional layer of complexity involves stereochemistry (the 3D arrangement of atoms) and protecting groups (temporary modifications used to “mask” reactive sites).
These elements function like temporary state variables or conditional flags that preserve critical information during intermediate steps.
For example, protecting groups are akin to placing certain variables in a “read-only” or “inactive” state to prevent unwanted changes until the program reaches the correct point in execution to “unprotect” them.
Stereochemical relationships act like metadata that must be preserved across transformations, ensuring that the final molecule has the correct 3D orientation—just as a compiler must preserve type information across optimizations.
\\
\indent
To manage these complex transformations systematically, chemists often model reactions in a linear data flow format:
\\
\begin{center}
{Reactants $+$ Reagents $\rightarrow$ Products}
\end{center}
\indent
This format mirrors data flow in functional programming or pipeline architectures in software engineering.
Each step is a transformation function that takes molecular “inputs” and produces “outputs,” which can then feed into the next step.
From a computational perspective, this linear structure offers several advantages:
\begin{enumerate}
    \item \textbf{Traceability:} Each transformation can be logged and indexed, enabling the reconstruction of synthetic routes and error checking, similar to how logs are used in debugging.
    \item \textbf{Database Integration:} Standardized formats like SMILES and SMIRKS allow reactions to be stored as structured data, facilitating search, retrieval, and machine learning. This mirrors how normalized database schemas support efficient querying.
    \item \textbf{Modularity:} Each reaction step is a modular operation that can be reused across different pathways, analogous to reusable functions or classes in code.
    \item \textbf{Visualization:} Linear reaction flows can be easily represented as directed acyclic graphs (DAGs), where nodes are molecular states and edges are transformations—making them ideal for cheminformatics pipelines and algorithmic reasoning.
\end{enumerate}
\indent
In short, multistep synthesis can be understood as a sophisticated algorithmic problem: molecules represent structured data, functional groups represent types and states, reactions are transformation functions, and retrosynthesis is a backward search strategy under complex constraints.
For computer scientists, this framing highlights why cheminformatics libraries like RDKit, symbolic languages like SMIRKS, and visualization pipelines are powerful—they translate the chemical logic into computational structures that can be analyzed, optimized, and taught using algorithmic principles.
\\

    
\section{Related Works}
\subsection{Reaction Representation and SMIRKS Formalism}
\indent
The foundational step in computational reaction modeling is the translation of chemical transformations into a symbolic language interpretable by algorithms.
The SMIRKS format, derived from SMILES, provides a flexible mechanism to encode both specific reactions and generalized transformation rules.
Literature on chemical data representation emphasizes that these formats enable reaction pattern recognition, rule-based synthesis prediction, and large-scale database integration.
Works in this domain highlight that robust SMIRKS parsing supports not only mechanistic modeling but also serves as a data layer for visualization systems and generative synthesis algorithms~\cite{10.1002/ail2.91}.
\\
\indent
Works in this domain highlight that robust SMIRKS parsing supports not only mechanistic modeling but also serves as a data layer for visualization systems and generative synthesis algorithms.
The annotated literature identifies several efforts to standardize chemical notation and link structural encoding with reaction prediction models, a crucial link to the project’s educational synthesis visualization goals.
\subsection{RDKit and Cheminformatics Frameworks}
\indent
Among open-source cheminformatics platforms, RDKit is one of the most widely adopted due to its robust handling of molecular representations and transformations. 
Numerous studies in cheminformatics emphasize RDKit’s versatility for tasks including substructure searching, molecular fingerprinting, and reaction enumeration.
Researchers have particularly noted its extensibility for educational and visualization purposes, allowing developers to generate 2D molecular depictions, reaction schemes, and datasets for computational learning systems.
\\
\indent
In this project, RDKit’s molecule parsing (MolFromSmiles) and coordinate generation (Compute2DCoords) functions serve as the computational backbone for converting textual reaction definitions into visual chemical diagrams.
By leveraging RDKit’s drawing modules (Draw.MolToImage), the project extends the toolkit into a visual pedagogy tool—making chemical synthesis pathways interpretable even for non-specialist learners.
This positions the work within a subfield of cheminformatics focused on interpretable visual analytics, bridging chemical informatics and visual communication.
\\
\subsection{Computational Visualization and Educational Tools}
\indent
The use of visualization in computational chemistry extends beyond aesthetics; it plays a vital cognitive role in supporting chemical reasoning.
Prior work on reaction visualization tools—ranging from commercial platforms like ChemDraw to research-oriented frameworks like Indigo and Open Babel—demonstrates how symbolic chemistry can be rendered graphically to assist comprehension and verification.
However, many of these systems rely on manual input or lack automated support for multi-step reaction sequences.
\\
\indent
By contrast, the system developed in this project automates pathway rendering from large reaction datasets. 
Each reaction, formatted as Reactants $\rightarrow$ Reagents $\rightarrow$ Products, is parsed and visually represented as a network of chemical structures connected by arrows annotated with reagent labels.
This approach aligns with modern pedagogical chemistry tools that emphasize interactivity and visual logic to improve understanding of synthesis design.
The use of PIL (Python Imaging Library) to dynamically compose RDKit-rendered molecules onto a unified canvas represents a practical integration of informatics and visual communication principles highlighted in the literature.
\\
\subsection{Data-Driven Synthesis Prediction and Multistep Modeling}
\indent
Recent studies in reaction prediction, retrosynthesis, and synthesis planning have leveraged large datasets encoded in SMILES/SMIRKS to train machine learning models capable of generating plausible synthetic routes~\cite{dai2019retrosynthesis}.
Although this project does not directly implement predictive modeling, it shares methodological continuity with such research by structuring reactions in a format compatible with machine learning frameworks~\cite{hormazabal2022cede}.
Literature examining data-driven synthesis design emphasizes that visualization and explainability remain key challenges; thus, tools that render reaction logic interpretable, like this project’s visualization pipeline, contribute to advancing both human understanding and computational transparency in chemistry.
\\
\indent
Moreover, by including thousands of curated textbook and large-scale reactions, the dataset architecture supports potential future expansion into reaction pathway decision trees or knowledge graph representations, which several annotated studies identify as emerging directions in cheminformatics research~\cite{10.1016/j.tcs.2025.115344}.

\section{Methodology}
Dijkstra's algorithm was chosen for single-source shortest path computation on graphs with non-negative edge weights, following the implementation presented in CLRS~\cite{clrsAlgorithms}.

\section{Results}

\section{Conclusion}

\bibliographystyle{acm}
\bibliography{bibliography.bib}

\end{document}
