% This is a template for your written document.
% To compile using latexmk on the command line, run the following: 
% latexmk -pdf main.tex

\documentclass[12pt]{article}
\usepackage{setspace}
\singlespace
\usepackage[left=1in,right=1in,top=1in,bottom=1in]{geometry}
\usepackage{graphicx}
\usepackage{float}

\title{\textbf{Multistep Synthesis Tool for Organic Chemistry Education}}
\author{Raelyn Brooks}

\begin{document}

\maketitle

\section{Introduction}
\indent
Organic synthesis—the design of reaction sequences to build complex molecules from simpler precursors—remains one of the most intellectually demanding areas of chemistry.
Unlike single-step reactions that can be memorized and applied in isolation, multistep synthesis requires strategic planning, backward reasoning, and the careful coordination of many interacting chemical transformations.
For students, this often feels like solving a "black box" puzzle; they understand individual reactions but struggle to combine them into coherent pathways.
The widespread difficulty in mastering this skill is underscored by national failure rates for organic chemistry courses, which frequently exceed 50 percent~\cite{mooring2016flipped}.
\\
\indent
In both research and education, the traditional static depiction of reactions in textbooks fails to reflect the underlying algorithmic logic of multistep synthesis.
Designing a synthetic route involves recursive problem solving (planning the route backward from the final goal), constraint management (ensuring chemical groups don't interfere with each other), and temporary state control.
These processes are functionally identical to complex computational planning problems.
However, existing instructional tools rarely make this rule-driven, step-by-step structure explicit, limiting learners' ability to see synthesis as a predictable, solvable system.
\\
\indent
This project addresses that critical gap by developing a visual mapping tool that translates textbook reactions into a dynamic, searchable network.
The tool functions by first encoding every molecule and reaction step using a specialized computer language developed for chemistry. 
It then uses professional cheminformatics software (a library of molecular drawing tools) to convert these coded definitions into high-quality visual diagrams. 
This process allows entire multistep pathways to be represented not as static images, but as connected, traceable flowcharts.
\\
\indent
By leveraging these computational methods, the project contributes to educational tools that automate the creation of synthesis maps from large datasets of real-world and textbook reactions. 
The resulting visualization is designed to emphasize the decision points and constraints inherent in chemical planning, effectively mirroring algorithmic reasoning. 
This work fundamentally bridges chemical and computational thinking, making synthesis more structured and accessible for both chemists and computer scientists. 
Ultimately, this work positions multistep synthesis not just as a chemical skill but as a computationally tractable process, paving the way for more effective teaching tools, database-driven route optimization, and algorithmic analysis in organic chemistry.
\\

\section{Background}
\indent
In organic chemistry, multistep synthesis refers to the process of transforming simple molecular starting materials into complex target molecules through a series of discrete, logically ordered chemical reactions.
Each reaction step modifies the structure of a molecule, typically by altering functional groups—specific arrangements of atoms that dictate chemical behavior.
From a computer science perspective, these functional groups can be viewed as data types or states: just as an integer might be cast into a string or an array into a list, a hydroxyl group (–OH) might be converted into a halide (–Cl), or an alkene might be transformed into an alcohol. 
Each synthetic step is thus a state transition in a well-defined, though highly constrained, chemical state machine.
\\
\indent
Designing a synthetic pathway is not unlike algorithm design, where chemists aim to find an efficient and feasible sequence of operations to reach a target state.
Here, the “output” is a desired molecule, and the “input” is a set of commercially available or easily accessible starting materials. 
The multistep aspect arises because direct, one-step conversions are rarely possible; instead, chemists must plan a sequence of reactions, each altering the molecular structure in a controlled manner.
This is analogous to chaining multiple functions in a program to transform an initial data structure into a final desired format.
\\
\indent
A central algorithmic strategy in synthesis planning is retrosynthesis, introduced by E.J. Corey, organic chemist and 1990 Chemistry Nobel Prize Winner.
Retrosynthesis involves working backward from the target molecule to break it down recursively into simpler precursors.
This is strikingly similar to goal decomposition in AI planning or backtracking algorithms in search problems.
Starting from the target, chemists iteratively identify strategic bonds to “disconnect,” yielding simpler molecules that could plausibly be converted into the target in one step.
This continues until reaching molecules that are known or easily obtainable starting materials.
In computational terms, retrosynthesis resembles a reverse search through a tree or graph, where each node represents a molecular state and each edge represents a known chemical transformation.
\\
\indent
During synthesis planning, selectivity and functional group compatibility act as constraints, analogous to conditional logic in code or constraint satisfaction problems (CSPs) in AI.~\cite{10.1109/ICTAI.2004.38}
For example, a reaction designed to modify an alcohol group might also inadvertently react with a nearby amine group, producing unwanted side products.
Chemists must therefore choose reaction sequences that respect these constraints, ensuring that each transformation occurs at the correct site and does not disrupt other parts of the molecule.
This is akin to managing function side effects in programming—ensuring that an operation modifies only what it is intended to, without corrupting unrelated data.
\\
\indent
An additional layer of complexity involves stereochemistry (the 3D arrangement of atoms) and protecting groups (temporary modifications used to “mask” reactive sites).
These elements function like temporary state variables or conditional flags that preserve critical information during intermediate steps.
For example, protecting groups are akin to placing certain variables in a “read-only” or “inactive” state to prevent unwanted changes until the program reaches the correct point in execution to “unprotect” them.
Stereochemical relationships act like metadata that must be preserved across transformations, ensuring that the final molecule has the correct 3D orientation—just as a compiler must preserve type information across optimizations.
\\
\indent
To manage these complex transformations systematically, chemists often model reactions in a linear data flow format:
\\
\begin{center}
{Reactants $+$ Reagents $\rightarrow$ Products}
\end{center}
\indent
This format mirrors data flow in functional programming or pipeline architectures in software engineering.
Each step is a transformation function that takes molecular “inputs” and produces “outputs,” which can then feed into the next step.
From a computational perspective, this linear structure offers several advantages:
\begin{enumerate}
    \item \textbf{Traceability:} Each transformation can be logged and indexed, enabling the reconstruction of synthetic routes and error checking, similar to how logs are used in debugging.
    \item \textbf{Database Integration:} Standardized formats like SMILES and SMIRKS allow reactions to be stored as structured data, facilitating search, retrieval, and machine learning. This mirrors how normalized database schemas support efficient querying.
    \item \textbf{Modularity:} Each reaction step is a modular operation that can be reused across different pathways, analogous to reusable functions or classes in code.
    \item \textbf{Visualization:} Linear reaction flows can be easily represented as directed acyclic graphs (DAGs), where nodes are molecular states and edges are transformations—making them ideal for cheminformatics pipelines and algorithmic reasoning.
\end{enumerate}

In short, multistep synthesis can be understood as a sophisticated algorithmic problem: molecules represent structured data, functional groups represent types and states, reactions are transformation functions, and retrosynthesis is a backward search strategy under complex constraints.
For computer scientists, this framing highlights why cheminformatics libraries like RDKit, symbolic languages like SMIRKS, and visualization pipelines are powerful—they translate the chemical logic into computational structures that can be analyzed, optimized, and taught using algorithmic principles.

\section{Related Works}
\subsection{Reaction Representation and SMIRKS Formalism}
\indent
The foundational step in computational reaction modeling is the translation of chemical transformations into a symbolic language interpretable by algorithms.
The SMIRKS (superset of SMILES) format, derived from SMILES (Simplified Molecular Input Line Entry System), provides a flexible mechanism to encode both specific reactions and generalized transformation rules.
Literature on chemical data representation emphasizes that these formats enable reaction pattern recognition, rule-based synthesis prediction, and large-scale database integration.
Works in this domain highlight that robust SMIRKS parsing supports not only mechanistic modeling but also serves as a data layer for visualization systems and generative synthesis algorithms~\cite{10.1002/ail2.91}.
\begin{figure}[H] % Suggests figure placement (here, top, bottom)
    \centering
    \includegraphics[scale=0.50]{SMILEStoMolecule.png}
    \caption{Visuals of SMILES notation to actual molecules}
    \label{fig:SMILEStomolecule} % Unique label for referencing
\end{figure}
These examples illustrates how SMILES strings are converted into 2D molecular structures using RDKit, demonstrating the practical application of symbolic chemical notation in computational chemistry.
The annotated literature identifies several efforts to standardize chemical notation and link structural encoding with reaction prediction models, a crucial link to the project’s educational synthesis visualization goals.
\subsection{RDKit and Cheminformatics Frameworks}
\indent
Among open-source cheminformatics platforms, RDKit is one of the most widely adopted due to its robust handling of molecular representations and transformations. 
Numerous studies in cheminformatics emphasize RDKit’s versatility for tasks including substructure searching, molecular fingerprinting, and reaction enumeration.
Researchers have particularly noted its extensibility for educational and visualization purposes, allowing developers to generate 2D molecular depictions, reaction schemes, and datasets for computational learning systems.
\\
In this project, RDKit’s molecule parsing (MolFromSmiles) and coordinate generation (Compute2DCoords) functions serve as the computational backbone for converting textual reaction definitions into visual chemical diagrams.
By leveraging RDKit’s drawing modules (Draw.MolToImage), the project extends the toolkit into a visual pedagogy tool—making chemical synthesis pathways interpretable even for non-specialist learners.
This positions the work within a subfield of cheminformatics focused on interpretable visual analytics, bridging chemical informatics and visual communication.
\\
\subsection{Computational Visualization and Educational Tools}

\noindent
The use of visualization in computational chemistry extends beyond aesthetics; it plays a vital cognitive role in supporting chemical reasoning.
Prior work on reaction visualization tools—ranging from commercial platforms like ChemDraw to research-oriented frameworks like Indigo and Open Babel—demonstrates how symbolic chemistry can be rendered graphically to assist comprehension and verification.
However, many of these systems rely on manual input or lack automated support for multi-step reaction sequences.
\\
\indent
By contrast, the system developed in this project automates pathway rendering from large reaction datasets. 
Each reaction, formatted as Reactants $\rightarrow$ Reagents $\rightarrow$ Products, is parsed and visually represented as a network of chemical structures connected by arrows annotated with reagent labels.
This approach aligns with modern pedagogical chemistry tools that emphasize interactivity and visual logic to improve understanding of synthesis design.
The use of PIL (Python Imaging Library) to dynamically compose RDKit-rendered molecules onto a unified canvas represents a practical integration of informatics and visual communication principles highlighted in the literature.
\\
\subsection{Data-Driven Synthesis Prediction and Multistep Modeling}
\indent
Recent studies in reaction prediction, retrosynthesis, and synthesis planning have leveraged large datasets encoded in SMILES/SMIRKS to train machine learning models capable of generating plausible synthetic routes~\cite{dai2019retrosynthesis}.
Although this project does not directly implement predictive modeling, it shares methodological continuity with such research by structuring reactions in a format compatible with machine learning frameworks~\cite{hormazabal2022cede}.
Literature examining data-driven synthesis design emphasizes that visualization and explainability remain key challenges; thus, tools that render reaction logic interpretable, like this project’s visualization pipeline, contribute to advancing both human understanding and computational transparency in chemistry.
\\
\indent
Moreover, by including thousands of curated textbook and large-scale reactions, the dataset architecture supports potential future expansion into reaction pathway decision trees or knowledge graph representations, which several annotated studies identify as emerging directions in cheminformatics research~\cite{10.1016/j.tcs.2025.115344}.
\subsection{Algorithmic and Graph-Based Modeling in Synthesis}
\noindent
The problem of synthesis planning is fundamentally an algorithmic search problem, mirroring goal decomposition in AI planning and backtracking algorithms in search problem~\cite{dai2019retrosynthesis}.
The seminal work of E.J. Corey established retrosynthesis as the dominant chemical strategy: a recursive backward search from the target molecule to simpler precursors.
Computationally, this process resembles a reverse search through a tree or graph where nodes are molecular states and edges are transformations.
This approach requires robust graph algorithms, such as Dijkstra's algorithm, to effectively determine the shortes or most economical path in terms of steps or cost~\cite{clrsAlgorithms}.
Furthermore, the challenges of selectivity and functional group compatibility in synthesis are analogous to constraint satisfaction problems (CSPs) in AI~\cite{10.1109/ICTAI.2004.38}.
Research has shown that decision trees can be effectively integrated with CSPs to manage constraints and optimize search efforts, providing a computational justification for linking chemical constraints to decision logic.
This focus on creating interpretable rules, through decision tree models directly supports the project's goal of visualizing decision points in the synthetic pathway~\cite{10.24963/ijcai.2023/225, 10.1016/j.tcs.2025.115344}.
\subsection{Computational Efficiency and Scalabilty}
\noindent
The design prioritized computational efficiency to ensure the system remained viable for processing large datasets and supporting a low-latency interactive user experience.
\begin{itemize}
    \item \textbf{Graph Construction:} The Reaction Graph utilized a Python \texttt{defaultdict(list)} to implement the adjacencey list. This structure allows for O(1) average time complexity for adding new reactions (edges) and looking up the neighbors (products) of any given reactant (node). This efficient data structure is crucial for managing the 1.37 million reaction entries without excessive overhead.
    \item \textbf{Path Finding:} As discussed, the system employs Breadth-First Search (BFS) on the reverse graph for optimal path discovery. In a graph with V molecules (nodes) and E reaction steps (edges), BFS operates in O(V + E) time. This linear time complexity ensures that even as the dataset scales, the search time remains predictable and fast, which is critical for supporting real-time user queries for synthesis pathways.
    \item \textbf{Molecular Canonicalization:} The use of 's canonical SMILES generation for every molecule ensures that each unique chemical compound maps to a single, consistent graph node. This eliminates redundant nodes and significantly reduces the size of V (the number of vertices), thereby directly improving the overall O(V + E) efficiency of the search algorithm.
\end{itemize}

\section{Methodology}
\subsection{Introduction to Methodology}
The design and implementation of the Multistep Synthesis Tool were guided by a core objective: to translate the complex, constraint-driven logic of organic synthesis into a computationally tractable and visually intuitive format.
This section details the data architecture, the graph-based framework for synthesis planning, and the cheminformatics visualization pipeline developed to achieve this goal.
\subsection{Data Acquisition and Structural Formatting}
The foundation of the system is a comprehensive dataset of chemical transformations.
The primary data source is a large SMILES dataset compiled by Rik van der Lingen, which encompasses 1.37 million reaction SMILES entries~\cite{vanderlingen2025reaction}.
These entries, which include reactants, reagents, solvents, and products, were sourced predominantly from the USPTO (United States Patent and Trademark Office) patent literature spanning from 1976 to 2024, with an additional 2.5 percent from academic literature.
This extensive dataset was generated through custom parsing and conversion from an existing USPTO dataset, employing OSCAR for semantic parsing and ChatGPT LLM-based extraction for data extraction.
Molecular entity identification was achieved using OPSIN and a custom synonym list.
All SMILES strings were verified to be RDKit-safe, and duplicates were removed to ensure data integrity.
\\
To complement this large-scale data, a custom collection of textbook organic chemistry reactions was curated and formatted explicitly in the SMIRKS notation.
The SMIRKS format is crucial for encoding chemical transformation rules, utilizing the convention: {Reactants $\rightarrow$ Reagents $\rightarrow$ Products}.
Within this structure, the "$>$" symbol separates the reaction components, while the "$.$" symbol is used to separate individual molecules within a component list.
This symbolic language enables the computational system to parse reaction logic, a necessary step for the visualization pipeline and future expansion into prediction models.
\begin{figure}[H] % Suggests figure placement (here, top, bottom)
    \centering
    \includegraphics[scale=0.35]{137DB.png}
    \caption{Snippet of the 1.37 million reaction SMILES dataset compiled by Rik van der Lingen, showcasing the extensive range of chemical transformations available for synthesis planning and visualization.}
    \label{fig:1.37MDB} % Unique label for referencing
\end{figure}

\subsection{Cheminformatics Visualization Pipeline}
The entire system was built using Python, leveraging key libraries for cheminformatics and image generation.
The development environment was maintained using Anaconda to ensure seamless integration with the RDKit environment~\cite{rdkit2025overview}.
\\
The pipeline comprises two main operational scripts and a structured data organization system:
\begin{enumerate}
    \item \textbf{combined123:} This core script is responsible for the actual rendering. It parses the SMIRKS notation by first splitting the reaction components using $">"$ and then individual molecules using $"."$. It uses the RDKit library for its robust handling of molecular representations and transformations,specifically for interpreting SMIRKS notation and generating 2D molecular diagrams.
    \item \textbf{generateFromDataset:} This script manages data access. It reads reaction SMILES and SMIRKS from the curated dataset folder, allowing for adjustable processing limits.
    \item \textbf{Progress Tracking:} It incorporates the tqdm library to provide runtime tracking and progress indication, crucial for managing the large data volume.
\end{enumerate}
The RDKit output (molecular structure images) is then composed using the Python Imaging Library (PIL) to create a unified canvas that displays the reaction scheme in the familiar reaction format.
All generated images (products, reactants, reactions, and complete multisteps) are stored in organized folders, facilitating rapid retrival for the interactive step-by-step user interface, which is designed with an input structure similar to RMechDB.
The ultimate goal of this pipeline is to provide a transparent, visual logic for chemical processes, advancing cheminformatics through interpretable visual analytics.
\subsection{Graph Representation and Unit-Weighted Search}
The synthesis planning problem was formally modeled as a directed acyclic graph (DAG), although the data processing does not explicitly enforce acyclicity in the raw data loading \texttt{reactionGraph.py}.
This structure, implemented using Python's \texttt{defaultdict(list)} as an adjacency list, is defined as follows:
\begin{enumerate}
    \item \textbf{Nodes:} Canonical SMILES strings representing all stable molecules (reactants, intermediates, products). Canonicalization is enforced by RDKit during parsing to ensure a single, consistent identity for each chemical species, which is crucial for accurate graph traversal and node uniqueness.
    \item \textbf{Edges:} Represented by the tuple (Product, Reagents), where the reagents are stored as an annotation on the edge.
\end{enumerate}
To find the shortest route, the synthesis was reframed as a unit-weighted shortest path problem.
Since the metric for optimization is the number of synthetic steps (edge count) rather than a complex cost function (like yield or time), all successful transformations are assigned a unit weight of 1.
Consequently, the optimal algorithm for path finding is the Breadth-First Search (BFS), as implemented in the \texttt{find\_path} method of the \texttt{reactionGraph} class.
BFS guarantees the discovery of the path with the minimum number of steps first, operating more efficiently than Dijkstra's algorithm in this specific context.
\subsection{Retrosynthetic Traversal Strategy}
To implement the retrosynthetic strategy proposed by Corey, the BFS is executed on a reverse graph structure.
While the primary graph tracks forward synthesis (Reactants $\rightarrow$ Products), a parallel \texttt{reverse\_graph} structure tracks Products $\rightarrow$ Reactants.
\\ 
The search begins by enqueuing the target molecule and explores backward using the reverse graph to identify the necessary precursors.
A deque is used for the BFS queue, and a visited set is maintained to prevent redundant exploration and cycles (where a reaction sequence leads back to a previously visited intermediate).
The search terminates when an intermediate molecule is found that exists within the predefined set of starting materials.
\\
The final pathway is then constructed by reversing the recorded sequence of steps (the path list in the BFS queue).
This strategy ensures the retrieved output is the desired forward synthesis pathway (Starting Material $\rightarrow$ Target Product) while maintaining the computational efficiency of the backward search approach.
\subsection{Implementation of Chemical Logic in SMIRKS Parsing}
The SMIRKS parser, implemented in the \texttt{combined123.py}, incorporates specific rules to translate the abstract SMIRKS notation into chemically meaningful steps.
The core parsing function is responsible for:
\begin{enumerate}
    \item \textbf{Component Separation:} Strict adherence to the Reactants $\rightarrow$ Reagents $\rightarrow$ Products convention is enforced using the '$>$' delimiter. The use of the '$ . $' delimiter allows for the identification of multiple molecular components (e.g., co-reactants or byproducts) within a single reaction step. 
    \item \textbf{Product Identification:} A critical challenge in automated parsing is identifying the single main product of interest, especially when side products are also encoded in the SMIRKS string. The system applies a chemical heuristic: it uses RDKIt to determine the main product by selecting the molecule with the largest number of heavy atoms. This pragmatic rule ensures that complex syntheses, which often produce minor or inorganic byproducts, are correctly simplified into the main transformation necessary for pathway generation.
    \item \textbf{Molecular Validation:} All parsed strings are immediately passed to RDKit's \texttt{MolFromSmiles} function. This serves as a vital data integrity check, ensuring that only syntactically valid and RDKit-compatible SMILES strings are converted into molecular objects and used to construct the reaction graph, filtering out potentially corrupt data from the large corpus.
\end{enumerate}

\section{Results}
\subsection{Successful Visualization of Elementary Transformations}
The first objective of this project—creating clear, annotated visualizations of chemical reactions—was achieved through the integration of the RDKit and PIL libraries, as detailed in the \texttt{combined123.py} script.
The \texttt{parse\_multistep\_smirks} function successfully converts SMIRKS notation (Reactants $\rightarrow$ Reagents $\rightarrow$ Products) into a structured list of steps, while the \texttt{draw\_multistep\_pathway} function rendered the chemical structures and reaction conditions.
\\
\indent
The system demonstrated its ability to visualize a basic, single-step transformation, such as the formation of a cyclic acetal:
\begin{figure}[H] % Suggests figure placement (here, top, bottom)
    \centering
    \includegraphics[scale=0.50]{example2.png}
    \caption{Visualization of a single-step reaction: Formation of a cyclic acetal from a diol and an aldehyde using an acid catalyst. The reactants, reagents, and product are clearly depicted, demonstrating the system's capability to render individual transformations.}
    \label{fig:cyclicacetal} % Unique label for referencing
\end{figure}
The system's reliance on RDKit also ensured that it could robustly handle molecules of significant size and complexity.
For instance, the system is capable of parsing and rendering structures like Aspirin CC(=O)Oc1ccccc1C(=O)O, demonstrating that the visualization pipeline scales effectively to handle the intricate molecular topologies commonly encountered in real-world organic synthesis research.
\subsection{Algorithmic Pathway Generation and Shortest Path Search}
The second key deliverable was the algorithmic identification of the shortest synthesis pathway.
This was implemented using the \texttt{reactionGraph} class (\texttt{reactionGraph.py}), which constructs a directed graph where nodes are molecules and edges represent reagent-driven transformations.
\\
\indent
The graph search implementation had to align with the chemical strategy of retrosynthesis, the \texttt{find\_path} method utilized a Breadth-First Search (BFS) algorithm on the reverse graph (where edges point (Products $\rightarrow$ Reactants)).
This approach implicitly models the problem as finding the minimum number of steps required to reach any available starting material from the target molecule, effectively providing the shortest synthesis route in terms of step count.
The BFS method guaranteed that the first path found was optimal for a unit-weighted graph, successfully addressing the need to return a concise and chemically reasonable route to the user.
\\
\indent
The system successfully processed the concatenated SMIRKS input (\texttt{CCO > H2SO4 > C=C; C=C > mCPBA > OCC; OCC > HBr > BrCC; BrCC > AgOAc > OCC}), generating a complete five-step synthetic pathway:
\begin{figure}[H] % Suggests figure placement (here, top, bottom)
    \centering
    \includegraphics[scale=0.25]{example3.png}
    \caption{Visualization of a multi-step synthesis pathway. The system successfully rendered a 5-step reaction sequence, demonstrating its capability to handle complex synthetic routes and provide clear visual logic for each transformation.}
    \label{fig:5steppathway} % Unique label for referencing
\end{figure}
\noindent
This visualization, generated by the combined pipeline, verified the system’s capacity to:
\begin{enumerate}
    \item Parse and link sequential reactions defined by SMIRKS notation.
    \item Track intermediate compounds (nodes) across steps.
    \item Display the complete pathway as a single, interpretable image, thereby transforming the abstract algorithmic search result into a concrete pedagogical tool.
\end{enumerate}
The use of an adjacency list structure (\texttt{defaultdict(list)}) in the \texttt{reactionGraph} further ensured that the graph building process from either the large dataset or curated textbook reactions was efficient and scalable, laying the groundwork for future expansions into larger reaction networks and more complex synthesis planning scenarios.

\section{Conclusion}
\noindent
This project successfully develops and implements an interactive visualization pipeline designed to demystify the algorithmic complexity of multistep organic synthesis for students and researchers.
By bridging the gap between chemical transformations and computational logic, the tool positioned synthesis not merely as a memorization task, but as a tractable, rule-driven problem.
\\
\indent
The foundation of the system relied on encoding chemical reaction rules using the SMIRKS formalism and leveraging the RDKit cheminformatics library for reliable structure rendering.
This choice enabled the translation of vast datasets, including over a million reaction SMILES entries sourced from USPTO patent literature, alongside a curated collection of textbook reactions, into machine-interpretable data structures.
\\
\indent
The core achievement was the establishment of a graph-based synthesis planning framework.
By modeling molecules as nodes and chemical transformations (reagents/conditions) as weighted edges, the system effectively captured the decision-tree structure inherent in synthesis planning.
The edge weights, defined by the number of synthetic steps, allowed the system to perform efficient path-finding, demonstrating the potential for O(1) lookup time for short, efficient routes.
This framing, which treats functional group compatibility as a form of constraint satisfaction and retrosynthesis as recursive backward reasoning, successfully converted the chemical challenge into a solvable algorithmic problem.
\\
\indent
The developed Python pipeline, utilizing RDKit and PIL, successfully parsed SMIRKS notation and automatically rendered high-quality, interpretable 2D images of both individual reaction steps and full multistep pathways.
This automated visualization capability provided a critical instructional aid, moving beyond static textbook depictions to offer a dynamic, traceable record of the synthetic logic.
\\
\indent
In conclusion, this work demonstrates the feasibility and power of applying computational methodologies to organic chemistry education.
By providing an explicit visual and algorithmic structure to multistep synthesis, the project laid the groundwork for future tools that could incorporate decision-tree logic and advanced AI models for retrosynthesis prediction, ultimately enhancing the teaching and practice of chemical synthesis.
\bibliographystyle{acm}
\bibliography{bibliography.bib}

\end{document}
